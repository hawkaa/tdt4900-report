\chapter{Genus App Platform}
\label{chap:gap}

This chapter contains a top-down analysis of \gap, with platform architecture, how applications are developed using the platform, concepts, and source code. The last part of this chapter identifies challenges in Genus App Platform, which motivates the second and primary part of this thesis, which is the design and experiment research.

We obtained information for this chapter by reading technical briefs, attending platform courses, engaging discussions with \genus' employees, and analyzing source code.


\clearpage
\section{Introduction}
\label{sec:Introduction}


\gap~is a platform for software development that aims to increase developer productivity and simplify change management through model-driven development \cite{Genus_AS2016-kt}. \gap~is a "no code" tool, which means end user applications are modeled without writing a single line of code. This approach enhances collaboration between business and IT, reduces life cycle costs, and lets the application developers focus on business logic instead of implementation details. 

\gap~uses generic software concepts on a higher level of abstraction than regular programming languages that are precise and non-ambiguous \cite{noauthor_undated-qy}. These concepts, and how they are implemented in the underlying platform, have been refined and improved through years of trial and error on real customer use cases.

If we relate \gap~to the model layers defined in Section \ref{sub:Models and Model Levels}, we see that it spans over three layers: \gap~platform developers work in the \textit{method engineering (ME) layer}, while expert users using the modeling tool to create end user applications work on the \textit{information system development (ISD) layer}. End users are in the \textit{information system (IS) layer}.

\section{Components and Architecture}
\label{sec:Components and Architecture}
\afigure{img/gap-components.png}{Components in \gap. \textit{Genus App Services} connects end user and modeling clients with the underlying database infrastructure, and provides authentication and session management. (Adapted from \cite{Genus_AS2016-kt})}{fig:gap-components}{0.7}

\gap~consists of five major components, as seen in Figure \ref{fig:gap-components}. These include two backend components, \textit{database infrastructure} and \textit{Genus App Services}, and one component for application modeling, \textit{Genus Studio}. \textit{Genus Desktop} and \textit{Genus Apps} are end user clients for \textit{Windows} and mobile devices respectively.

\subsection{Database Infrastructure}
\label{sub:Database Infrastructure}

The database infrastructure stores the application data, as well as the application itself (application model). \gap~connects to a wide variety of data providers, often traditional relational databases such as \pn{Oracle Database}, \pn{Microsoft SQL Server}, and \pn{MySQL}. However, there exists several adapters, or object-relational mappers, for other data sources, including SOAP-based web services. More integrations are planned in the future, which include adapters for NoSQL databases and RESTful web APIs. 

Applications developed in \gap~may build on already existing data sources or define a new, or partially new, data source to store application data. Since \gap~handles all data integrity with validations, triggers, duplicate handling, and foreign keys, such constraints do not need to be specified and defined in the database infrastructure.


\subsection{Genus App Services}
\label{sub:Genus App Services}
\pn{Genus App Services} is the main backend component in \gap. It facilitates communication between the clients and the database infrastructure, handles authentication, and manages sessions. \pn{Genus App Services} is made up of several nodes and node groups, where each node in a node group is configured to run one or more services that are required by the application. Requests to \pn{Genus App Services} are sent to the right node group based on application model and application dataset, and to the correct node based on which service is required. 

All nodes in \pn{Genus App Services} are stateless. This principle simplifies horizontal scaling since more nodes may be added if one particular service becomes a bottleneck. Nodes may still use caching mechanisms for performance reasons.

\subsection{Genus Studio}
\label{sub:Genus Studio}
\pn{Genus Studio} is the software modeling tool in \gap. In this tool, expert users use models close to the business problem and generic software concepts to create end user applications. In \mde~terminology, this tool belongs to the information system development layer. We study application development in \textit{Genus Studio} in Section \ref{sec:Application Development}.

\textit{Genus Studio} runs on the \textit{Windows} platform.

\subsection{Genus Desktop and Genus Apps}
\label{sub:Genus Desktop and Genus Apps}
\textit{Genus Desktop} and \textit{Genus Apps} are the end user clients for \textit{Windows} and mobile platforms respectively. End users access their applications and their respective data using these clients. Instances of these clients point to an application model and a corresponding dataset, and they communicate with \textit{Genus App Services} to receive sufficient model information and data to run the application. 

In \mdd~terminology, \textit{Genus Desktop} and \textit{Genus Apps} correspond to the \textit{Information System Layer}.

\section{Application Development}
\label{sec:Application Development}
\textit{Genus Studio} is the software modeling tool in \gap~and is used by expert users to design and develop end user applications. Applications are defined by specifying three different layers: The \textit{data layer}, \textit{logic layer}, and \textit{user interface layer}. These layers are analogous to the information structure, microflow, and form models used by \mendix.

\subsection{Data Layer}
\label{sub:Data Layer}
\afigure{img/gap-data-layer.png}{The data layer in \gap. This layer defines object classes, object relationships, data integrity and calculation, and security. (Adapted from \cite{Genus_AS2016-kt})}{fig:gap-data-layer}{0.8}

One of the layers in \gap~application development is the \textit{data layer}, which is seen in Figure \ref{fig:gap-data-layer}. This layer contains object class definitions with properties, as well as hierarchies and connections to explain how object classes relate to each other. The data layer specifies how data is fetched and stored in the underlying database infrastructure, for instance by specifying database connection strings and table names. Schemas in the data layer represents how the object model is represented when exposed through an API, which is primarily used to create and consume SOAP-based web services.

The data layer provides security functionality. This feature limits access to certain objects or certain properties. Other core functionality specified in the data layer include integrity and validity checks, formulas, triggers, and events.

A class diagram is used to model the data layer, which we will discuss in greater detail in Section \ref{sec:Concepts}.

\subsection{Logic Layer}
\label{sub:Logic Layer}
% Logic layer: Vi kaller byggeklossene for Effects (siden de har «side»-effekter) og prosessen med å sette dem sammen for Action Orchestration.

% skriv om denne etterpå
\afigure{img/gap-logic-layer.png}{The logic layer in \gap. Here, tasks and effects are specified through action orchestration. Tasks contain programming-like constructs and is composed of a wide variety of actions. (Adapted from \cite{Genus_AS2016-kt})}{fig:gap-logic-layer}{1.0}

The \textit{logic layer} in \gap~lets the modeler extend the application with custom logic. The main components in this layer are \textit{tasks and effects}, which is analogous to the microflows in \mendix.  The logic layer is shown in \ref{fig:gap-logic-layer}.

Effects and tasks are created through \textit{Action Orchestration}, which lets the system developer specify logic through programming language-like constructs, such as loops, conditional statements, and exceptions. There is a large number of different actions that might be included in an orchestration, which include object creation, modification, and delete, file handling, and consuming web services. There are several ways to run an effect; for instance through clicks and interaction in the user interface, recurring events triggered by a timer, or triggers caused by changes in the data.

\subsection{User Interface Layer}
\label{sub:User Interface Layer}
\afigure{img/gap-user-interface-layer.png}{The user interface layer in \gap. In this layer, forms and tables are specified, with a wide variety of available GUI elements. (Adapted from \cite{Genus_AS2016-kt})}{fig:gap-user-interface-layer}{1.0}

The GUI made available to the end users is modeled in the \textit{user interface layer}. The main components in this layer are the table and form views, where the latter can be designed for either \textit{Genus Desktop} or \textit{Genus Apps}. Form views allow the modeler to specify custom layouts and include a wide variety of different GUI elements. Supported elements, which is seen in Figure \ref{fig:gap-user-interface-layer}, include text boxes, buttons, calendars, maps, and more.

Components in the user interface layer fetch data from a data source, which specifies which data that should be made available to the users. The user interface may interact with the logic layer through user interaction, such as button clicks.

\section{Concepts}
\label{sec:Concepts}
To better understand how the three application development layers relate to \gap~source code, we elaborate on some important concepts used in the platform.

\subsection{Object Classes}
\label{sub:Object Classes}
\afigure{img/gap-class-diagram.png}{Class diagram from \textit{Genus Studio} for the \tpch. Blue units represent object domain compositions which mean they are mapped to the underlying database infrastructure. White units represent code domain compositions where values are "hard coded" and fetched from the application model itself.}{fig:gap-class-diagram}{1.0}
An object class in \gap~is a hybrid between a table definition in a relational database and a class in a programming language. An object class, sometimes referred to a composition, specifies several class properties (Section \ref{sub:Object Class Properties}), security settings, integrity constraints, formulas, and other display options. An object class can either be in the \textit{object domain} or the \textit{code domain}. Compositions belonging to the object domain must specify how it relates to the underlying database infrastructure, for instance by providing a database connection string and table name, or an XML schema if the data is fetched using a SOAP-based web service. Object classes in the code domain do not have this ability, since these values are "hard coded" into the application model, and not fetched from any data source.

Object classes are defined in the application data layer and specified in a class diagram, which is depicted in Figure \ref{fig:gap-class-diagram}.


\subsection{Object Class Properties}
\label{sub:Object Class Properties}
An object class is composed of one or more properties. Such properties can be simple data types, like integers and strings, or a function type, which calculates their value based on other properties in the class. In addition to specifying the basic data type that holds the value, properties also define an interpretation of the data. There exists many such interpretations in \gap, including \textit{file}, \textit{password}, \textit{date format}, and \textit{color}. The data interpretation may also be another object class, which is used on foreign key fields. This creates relationships between object classes in the data layer.

An object class property has its own set of security and validation rules and includes display options like formatting and screen tip.

\subsection{Data Source}
\label{sub:Data Source}
A \textit{data source} in \gap~is used by effects, tables, and forms to retrieve, create, modify, and delete data. One particular data source is associated with an object class, a filter which defines which subset of the objects that should be available, and cardinality; whether there may be one or multiple elements in the data source. A data source stores the composition objects in main memory of the clients.

Which data subset that is loaded into a data source is specified using a filter. This filter is translated to SQL or another relevant query language and then sent to the database infrastructure. Thus, at the initial data load, \gap~utilizes efficient database servers and reduce network traffic as data is filtered in the backed. However, a data source does not need to load any data at initialization. Empty data sources are, for instance, used for temporary in-memory processing.

\subsection{Analysis}
\label{sub:Analysis}
\afigure{img/gap-analysis.png}{\gap~analysis, or \gd, user interface.}{fig:genus-discovery}{1.0}
The \textit{analysis} component, sometimes refered to as \textit{Genus Discovery}, is the \bd~functionality in \gap. \textit{Genus Discovery} is is very similar to \qlikview, which we studied in \ref{sec:Business Discovery}, but since object classes and relations are already specified in the data layer, no data import script is needed. Like \qlikview, the analysis component allows end users to follow their "information scent" and click their way through the data using an intuitive user interface. An example user interface is seen in Figure \ref{fig:genus-discovery}.

One particular dashboard is referred to as \textit{an analysis}. Each analysis links to a specific data extract, or \textit{data mart}, which we study in the next section.

\subsection{Data Mart}
\label{sub:Data Mart}
\afigure{img/gap-data-mart.png}{Data mart for the \tpchdl, which contains only the object classes and fields needed to answer that particular query. The mart conforms to a snowfalke schema.}{fig:gap-data-mart}{0.7}

A data mart defines a data extract that is used by the analysis component in \gap. A data mart is specified by a set of data sources, and for each data source, a list of published fields. All data sources in a mart might specify a filter, such that the analyses only access a particular subset of the data. 

The process to define a data mart, including the user interface, is similar to defining object classes in an object diagram, like seen in Figure \ref{fig:gap-data-mart}. However, there is one important distinction: A data mart must conform to a snowflake schema, which means there can only exist one path through the class diagram. For instance, in Figure \ref{fig:gap-class-diagram}, both the customer and supplier classes refer to the nation class. In a data mart, there would be two nation entities; one entity for customer nationality and one for supplier nationality.

\subsection{Tasks}
\label{sub:Tasks}

\afigure{img/gap-task.png}{A task definition in \gap. A task is composed of actions and effects. This way, a task may read from and to data sources, enumerate items using for or while loops, consume services, import, export, and more.}{fig:gap-task}{1.0}

Tasks in \gap~are similar to a procedure or a function in a programming language, and is defined at the logic layer. A custom user interface, depicted in Figure \ref{fig:gap-task}, enables modelers to combine actions and effects into logic, a process denoted as \textit{action orchestration}. Tasks use data sources, which we saw in Section \ref{sub:Data Source}, to retrieve, create, modify, or delete data. Tasks might be triggered manually, by user interface interaction, or periodically by \gap~agents.

\section{Source Code and Classes}
\label{sec:Source Code and Classes}
\afigure{img/gap-original-class-diagram.png}{A class diagram with the most important classes in \gap~data representation.}{fig:gap-original-class-diagram}{0.9}
This section aims to give an overview over the classes used to represent data in \gap, which includes the representation of object classes, object class properties, object instances, and data elements. Figure \ref{fig:gap-original-class-diagram} displays a class diagram for the classes we discuss in this section.

The source code of \gap~core is written in \delphi, which we studied in Section \ref{sec:Delphi Programming Language}.

\subsection{GValue}
\label{sub:GValue}
The \cn{GValue} class, short for \textit{Genus Value}, is a base class that can contain any value of any type that is supported by the platform. These types include integers, floating point numbers, strings, object handles, dates, and more. The \cn{GValue} class is a necessity in \gap~to omit the rigid type system in \delphi, and to enable that all values, irrespective of type, are handled the same way.

For all value types supported in \gap, there are get and set methods for that specific value type, which by default throw exceptions indicating that they are not implemented. Then, for each type, a subclass exists which stores the data and overrides the corresponding get and set methods. For instance, there exists an \cn{IntegerValue} subclass with an integer property that extends \cn{GValue} and overrides the \cn{GetAsInteger} and \cn{SetAsInteger} methods. 

% memory handling
A \cn{GValue} is a class type and not a record type, hence is allocated by the memory manager and put on the heap. Thus, a \cn{GValue} variable is a pointer. Like strings and dynamic arrays, \cn{GValue} has built-in reference counting to enable cloning and data re-use. Still, the class is a true value type, which implies that it is immutable.

\subsection{CompositionDescriptor}
\label{sub:CompositionDescriptor}
The object class, or composition, which we discussed in Section \ref{sub:Object Classes}, is represented by the \cn{CompositionDescriptor} class. Conceptually, this class is the same as a class in a programming language. A \cn{CompositionDescriptor} contains all information needed to describe an object class, for instance which properties, or \cn{FieldDescriptor}s, that belongs to to the composition.

Since \cn{CompositionDescriptor} instances are a part of the application model, they belong to the ISD layer, which we discussed in Section \ref{sub:Models and Model Levels}. In the linguistic meta modeling framework, this construct makes up the left part of Figure \ref{fig:linguistic-metamodeling}, which is the "Class" and "Collie" entities.

\subsection{CompositionObject}
\label{sub:CompositionObject}
Whereas the \cn{CompositionDescriptor} class represents an object class definition, the \cn{CompositionObject} class represents particular instances of that object class. Conceptually, this class is the same as an object, or class instance, in a standard object oriented programming language. A \cn{CompositionObject} stores all data belonging to the particular instance in a \cn{CompositionObjectValueCollection}, as well as some other object state variables.

Instances of \cn{CompositionObject}s make up the data in the applicaiton, and is, therefore, associated with the IS layer in \mde. In the linguistic meta modeling framework, this class is depicted on the right part of Figure \ref{fig:linguistic-metamodeling}. \cn{CompositionObject} has an ontological \textit{instance-of} relationship with \cn{CompositionDescriptor}.

\subsection{FieldDescriptor}
\label{sub:FieldDescriptor}
The \cn{FieldDescriptor} class describes an attribute in a composition, which is the object class property we studied in Section \ref{sub:Object Class Properties}. It holds all relevant data about a property, mainly type and representation, but also constraints and formatting rules. Like \cn{CompositionDescriptor}, instances of the \cn{FieldDescriptor} class exist in the ISD layer.

\subsection{CompositionObjectValueCollection}
\label{sub:CompositionObjectValueCollection}
The \cn{CompositionObjectValueCollection} class holds all data for a composition object, and has a one-to-one relationship with the \cn{CompositionObject} class. Conceptually, this class represents all the properties with corresponding data stored in a class. Like a \cn{CompositionObject}, instatiations of this class belongs in the IS layer.

A \cn{CompositionObjectValueCollection} contains two lists, one with object class properties, and one with \cn{GValues}. To look up a particular property, a linear search through the property list is performed, and when the particular property is found, the function returns the corresponding \cn{GValue}. One way to see the \cn{CompositionObjectValueCollection} class, is as a database table with one row.


\section{Challenges in Genus App Platform}
\label{sec:Challenges in Genus App Platform}

There are several challenges in the current \gap~design. For years, the main area of focus has been to keep the source code readable and maintainable. However, this approach does not normally lead to an implementation that is optimal performance-wise. In this section, we discuss some parts of the \gap~source code that is not optimal and has resulted in challenges related to high memory usage and poor memory access patterns.

\subsection{Excessive Amount of Pointers}
\label{sub:Excessive Amount of Pointers}
\afigure{img/gap-original-instantiation.png}{A data source with three composition objects, where each object has two properties. For each object, there are pointers to both data values and field descriptors.}{fig:gap-original-instantiation}{1.9}

One of the main challenges in \gap~is that a significant amount of pointers is needed to store object data. As we saw in Section \ref{sub:CompositionObjectValueCollection}, each \cn{CompositionObjectValueCollection} not only stores \cn{GValue}s containing the data, but also pointers to all data descriptors. Although this implementation is flexible because every value collection is self-contained and objects can have a variable number of fields loaded, the overhead for storing all these pointers is substantial, especially on a 64-bit architecture where all pointers are 8 bytes. Figure \ref{fig:gap-original-instantiation} illustrates all pointers in play when a data source is filled with data. 

We study the storage overhead caused by pointers using a simple example. In our example, we use an object class with 15 object class properties. In the data source, there are 1,000,000 elements. The pointers needed in this setup are:
\begin{itemize}
    \item \textbf{1 pointer}, for the data source.
    \item \textbf{2 pointers}, for the composition descriptor and the pointer to the list of data descriptors.
    \item \textbf{15 pointers}, for the composition data descriptors
    \item Then, for every object:
    \begin{itemize}
        \item \textbf{1 pointer}, for the composition object.
        \item \textbf{1 pointer}, for the value collection pointer.
        \item \textbf{15 pointers}, for all data descriptors in the value collection.
        \item \textbf{15 pointers}, for all values in the value collection.
    \end{itemize}
\end{itemize}
The example results in 32,000,018 pointers, which is roughly 244 MB on a 64-bit architecture. Moreover, this is only the overhead caused by pointers, not the actual data. If all values are 32-bit integers, the memory needed store these 15 million numbers is 57 MB, which in this example is 20 \% of the space used by pointers. For 1 byte booleans, the percentage is even less.

\subsection{Inefficient Data Access and Poor Memory Locality}
\label{sub:Inefficient Data Access and Poor Memory Locality}
Several operations in \gap~access many values in tight loops, and it is not uncommon they read all values for a given data descriptor in a data source. Examples of such operations are filter and join operations. However, accessing values in the current implementation comes with a substantial overhead. The following steps are performed for every data access:
\begin{itemize}
    \item The \cn{CompositionObjectValueCollection} is scanned linearly for the correct data descriptor.
    \item The corresponding \cn{GValue} is found, and the accessor method (i.e. \fn{GetAsInteger}) is looked up in the dispatch table.
    \item A new stack frame is put on the stack to run the \fn{GetAsInteger} method.
\end{itemize}

The compiler might be able to optimize some of the steps above. For instance, if the \textit{register} calling convention is used, values may be accessed without creating a new stack frame. However, the process of getting a value is far from ideal. As we saw in Section \ref{sec:Hardware Utilization}, the difference between minimal and maximal CPU utilization can easily be one order of magnitude, where branches and new stack frames may severely hurt performance. Last, the linear search through data descriptors hurt throughput.

Not only are there many steps required to access data in \gap, but there is also poor memory locality on the data elements. Since \cn{GValue} is a class type, thus allocated on the heap by the memory manager, we no longer have control over where values are located in main memory. We must assume the location is arbitrary, and that \cn{GValue}s with temporal locality are not located next to each other in memory. We read in Chapter \ref{chap:olap} how important cache locality is for performance.

\subsection{Storage Overhead}
\label{sub:Storage Overhead}
\cn{GValue}s have much storage overhead. First, since they are reference counted, each value contains a 4-byte integer. Second, since a \cn{GValue} is a class type, the first 8 bytes in an object is a pointer to a virtual method table. Last, a \cn{GValue} variable is a pointer to the structure containing the data, a pointer which is an additional 8 bytes. Hence, for a 32-bit integer, only 4 of 24 bytes are used to store the actual data. 

Secondly, \gap~applies no form of compression. Using the same reasoning as above, we see that each boolean value will require 21 bytes to store. Theoretically, only 1 bit is needed to store such value, and we studied in the section about compression, Section \ref{sec:Compression}, how to achieve this effect. Reducing the memory per value from 21 bytes to 1 bit is a memory reduction of almost 200 times.

We saw in Chapter \ref{chap:olap} that small storage overhead and compression are paramount for read-optimized databases. Efficient data storage may turn a process from being I/O bound to CPU bound.

\section{Research Motivation}
\label{sec:Research Motivation}
This chapter concludes the related work part of our research. We have studied background theory on \mde~and \bi, elaborated on techniques used by read-optimized databases, and provided an analysis of \gap. The literature review has given us sufficient insight to proceed with the evaluation part of this research. 

We observe that \gap~is in fact a sophisticated in-memory database. Data sources used in various parts of the platform query the underlying database infrastructure and loads the data into main memory. We also see \gap~source code has several challenges with pointer and storage overhead and data locality. We repeat our research goals:

\setlength{\leftskip}{1cm}

\textbf{G1: Reduce memory consumption in \gap~and increase the platform's ability to handle and analyze large datasets.}

\textbf{G2: Introduce new evidence that \mde~can benefit from in-memory database technologies.}

\setlength{\leftskip}{0pt}

We are motivated to apply techniques used in read-optimized databases to \gap. We believe techniques like column storage and compression can mitigate several problems in \gap~source code, such that memory usage can be reduced, and the platform gains the ability to handle and analyze large datasets. We also believe this approach serves an example of how such problems are solved in the context of \mde. As a result of this, we form our research question:

\setlength{\leftskip}{1cm}

\textbf{RQ1: How can technology used by in-memory, read-optimized databases improve \gap's ability to handle and analyze large datasets, and what can model-driven engineering, in general, learn from database technology?}

\setlength{\leftskip}{0pt}

By answering \textbf{RQ1}, we hope to address \textbf{G1} directly by making changes in \gap~that increase the platform's ability to handle large datasets. However, by using our changes in \gap~as a proof-of-concept, we plan to address \textbf{G2} by drawing general conclusions on the combination of \mde~and database technology.

We plan to optimize \gap~in the method engineering layer, such that the information system development layer is affected as little as possible. Much like a compiler is better at optimizing code than a human programmer, we believe \gap~has the potential to optimize storage format without modeler intervention. This way, \gap~expert users may focus on the business problem and not on the underlying implementation. 

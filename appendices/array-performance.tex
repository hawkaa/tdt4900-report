\chapter{Array and Data Type Performance in Delphi}
\label{app:array-performance}

\section{Introduction}
\label{sec:Introduction}

In this microbenchmark, we test the performance and memory usage of different numeric data types in \delphi. We also study the performance of various array types found in the programming language and the standard library. We will assess \cn{integer}, \cn{Int64}, and \cn{double} data types and \cn{array of *}, \cn{TArray}, and \cn{TList} array types.

The motivation for this microbenchmark is to find the most efficient data types for column storage in \gap, both regarding memory usage and performance.

\begin{delphicode}{Array Performance Benchmark}{lst:array-performance}
  SetLength(l_arrNumbers, 4000000 * 10);
  for l_i := 1 to 4000000 do
    for l_j := 0 to 9 do
      l_arrNumbers[l_i * 10 + l_j] := l_j;
  l_oMeasurer.After('List', 4000000*10);

  for l_i := 1 to 10 do
  begin
    l_dSum := 0.0;
    l_oTimer.Start('Sum ' + IntToStr(l_i));
    for l_j := 0 to (4000000 * 10) - 1 do
      l_dSum := l_dSum + l_arrNumbers[l_j];
    l_oTimer.Stop('Sum ' + IntToStr(l_i));
  end;
\end{delphicode}

\section{Test Setup}
\label{sec:Test Setup}

For this microbenchmark, we write a small \delphi~program that fills an array structure of choice with numbers from 0 to 9 $n$ times. Then, we iterate the numbers and accumulate the sum into an accumulator variable. Listing \ref{lst:array-performance} shows the program.

We measure memory usage before and after allocation and array population. Then, we measure the average time it takes to sum all numbers in the array. We choose $n = 4000000$ for this benchmark.

The test is run on a Windows 10 Dell workstation with a 64-bit, 2.40 GHz Intel\textregistered Xeon\textregistered E5620 processor, and 8.00 GB RAM. The \delphi~version used is 23.0.21418.4207. We query the built-in \name{FastMM} memory manager to measure memory.

\section{Results}
\label{sec:Results}

\begin{table}
    \centering
        \begin{tabularx}{0.75\textwidth}{X | X | X | X}
        & \cn{integer} & \cn{Int64} & \cn{double} \\
        \hline
        \hline
        \cn{array of} & $4.00$ bytes & $8.00$ bytes & $8.00$ bytes \\
        \cn{TArray} & $4.00$ bytes & $8.00$ bytes & $8.00$ bytes \\
        \cn{TList} & $4.00$ bytes & $8.00$ bytes & $8.00$ bytes \\
    \end{tabularx}
    \caption{The memory used per element for different data types (columns) and array types (rows).}
    \label{tab:array-performance-memory}
\end{table}

As seen in Table \ref{tab:array-performance-memory}, all array implementations use equally many bytes per element, which means any overhead associated with these array types is negligible. As expected, the \cn{integer} type takes 4 bytes, while \cn{Int64} and \cn{double} takes 8 bytes.

\begin{table}
    \centering
    \begin{tabularx}{0.75\textwidth}{X | X | X | X}
        & \cn{integer} & \cn{Int64} & \cn{double} \\
        \hline
        \hline
        \cn{array of} & 131 ms & 134 ms & 157 ms\\
        \cn{TArray} & 130 ms & 134 ms & 157 ms\\
        \cn{TList} & 215 ms & 221 ms & 237 ms\\
    \end{tabularx}
    \caption{The average time it takes to sum all numbers in the benchmark for different data types (columns) and array types (rows).}
    \label{tab:array-performance}
\end{table}

The results of the performance test are shown in Table \ref{tab:array-performance}. We see that \cn{array of} and \cn{TArray} types are equal regarding performance. \cn{TList} is 1.5 - 2 times slower than these types. Also, integer addition is faster than floating point addition. There is not a significant difference in performance between 32-bit and 64-bit integer types.

\section{Conclusion}
\label{sec:Conclusion}

We conclude that either \cn{array of} or \cn{TArray} is better suited to store column data than \cn{TList}. In addition, the \cn{integer} or \cn{Int64} data type should be used over \cn{double} whenever possible.

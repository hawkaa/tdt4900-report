\chapter{Introduction}
\label{chap:introduction}

\section{Background and Motivation}
\label{sec:Background and Motivation}
% Paragraph about database technology and OLAP solutions.
Database technology have had much progress in the last couple of years. The traditional OLTP systems with row stores are being supplemented with high-performance column stores for quick data analysis. Not only does column store improve performance aggregation performance, but it also reduces memory footprint as aggressive compression can be applied.

% Paragraph about MDD
\mdd~has gained much more traction recently. Although one may think that an extra level of abstraction will reduce overall performance, this technique allows for highly optimized implementations underneath. 

% Paragraph about the intersection of these two fields.
Database technology and column store is one such implementation that can be changed out and applied to the whole framework. We therefore implement column store within \genusSoftware to see which aspects of this technique are beneficial for \mdd.

% Paragraph of what we will investigate (GENUS specific)

% Paragraph about how we got the idea to the research
Our original research was focusing on business discovery optimalizations. However, we came up with the idea to use the techniques learnt from the database techonolgy dircectly as base structures within \genusSoftware.

\section{Problem Statement, Goals, and Deliverables}
\label{sec:Problem Statement, Goals, and Deliverables}
Based on our motivation, we define the goal of the project as following: 

\textbf{G1: Improve performance and reduce memory consumption in \genusSoftware~for all the major use cases in the framework by applying state-of-the-art database technology in the core.}

We focus primarily on three major tasks:
\begin{itemize}
    \item \bd~capabilities.
    \item Batch operations.
    \item Filtering tasks.
\end{itemize}

We define \textit{state-of-the-art etc.} as following:
\begin{itemize}
    \item It's super new and the best!
\end{itemize}

We have defined this goal because we know that for certain tasks \genusSoftware~are too slow and use too much memory.

As a step towards reaching \textbf{G1}, we address the following research question:

\textbf{RQ1: Which benefits will a column store give a model driven development framework?} 

For \textbf{RQ1}, not only do we study the benefits, we also try to define a set of operations that can be used by other parts of the application to create efficient code.

\textbf{RQ2: Which metadata should be used to decide on the storage format?} 

\section{Methodology}
\label{sec:Methodology}
We implement column store in \genusSoftware and measure every change we do to get an overview over what affects the total performance.

\section{Deliverables and Contributions}
\label{sec:Deliverables and Contributions}
The main deliverable from this research is this report containing implementation details and test results.

Our contributions of this research are two-fold. First, we plan on \textit{introduce new evidence} for the intersection of \mdd~and database technology, something we claim has not been done before. We see several papers on \mdd, but none of them have explicitly studied the storage format within such systems.

Secondly, we contribute to an \textit{improved computer-based product}. We hope the users of \genusSoftware~will be excited about the results.

\section{Definitions}
\label{sec:Definitions}
This section presents terms, definitions, and important products that are relevant to understanding the content of this report.

\paragraph{Online Analytical Processing (OLAP)}
\label{par:Online Analytical Processing (OLAP)}
  We use the term Online Analytical Processing (OLAP) extensively in this report. By OLAP, we mean systems that enable users to analyze multidimensional data interactively from multiple perspectives \cite{Wikipedia_contributors2015-hw}. OLAP is usually dominated by ad-hoc, complex queries that group, aggregate and summarize over large datasets \cite{Bjorklund2011-wh}. OLAP systems can be both disk and memory based. Column storage is considered to be an attractive solution for OLAP systems, a technique we study further in Chapter \ref{chap:Data Layout}.


\paragraph{Online Transactional Processing (OLTP)}
\label{par:Online Transactional Processing (OLTP)}
Online Transactional Processing (OLTP) is a class of database systems that manage transaction-oriented applications \cite{Wikipedia_contributors2015-cw}. Transactional workloads are typically referred to as insertion of new records, as well as updates and deletes of single records in the database. An OLTP system normally uses row storage for its data.

\paragraph{Database Management System (DBMS)}
\label{par:Database Management System (DBMS)}
A Database Management System (DBMS) is a computer software application for storage and analysis of data \cite{Wikipedia_contributors2015-pb}. The most common way to interface with a database is through SQL, although other methods exist. Regarding performance, DBMSes can focus on analytical workloads (OLAP), transactional performance (OLTP), or both. DMBSes do not come with user interfaces for \bd~but is designed such that other applications can be built on top of them. In this report, we look at \oracle, \ibm, \saph, \sapnw, \mssql, \cstore, \vertica, \blink, \exasol, \oracle, \hyper, and \hyrise.

\paragraph{\bd}
\label{par:Business Discovery}
\bd~is a term introduced by a \qlikview~whitepaper \cite{Qlik2014-vd}. \bd~products differ from traditional \bi~systems by focusing more on the end user. \bd~products do not rely on aggregated data such that the user can follow his "information scent" and click his way through the data. \bd~platforms often provide an architecture that enables panels and dashboards to be shared with multiple clients, both on desktops and mobile devices. Current \bd~products typically build on tailored storage systems that are specifically designed for \bd~workloads, but some of them integrate directly with read-optimized DBMSes. \bd~products include \tableau, \qlikview, \powerpivot, and more. \bd~is explained in greater detail is Section~\ref{sec:Business Discovery}.

\paragraph{Reference products}
\label{par:Reference products}
We will occasionally use the term \textit{reference products} in this report. By reference products we mean \bd~products pointed out by \genus~that \bd~capabilities in \genusSoftware~will be compared to. In this report, we study \qlikview~and \tableau.


\section{Report Outline}
\label{sec:Report Outline}




\chapter{Introduction}
\label{chap:introduction}
In this chapter we explain the background information and motivation of this research project, and based on this, we state our goals and research questions. We then proceed to explain how we plan to answer the research questions by stating our methodology, deliverables, and contributions.
\clearpage

\section{Background and Motivation}
\label{sec:Background and Motivation}
RAM is getting cheaper \cite{Exasol2014-xh}, and together with 64-bits CPUs, in-memory databases are getting an increasingly more significant role \cite{Delaney2014-ip}. Systems capable of using main memory as primary storage include \oracle~\cite{Lahiri2015-mz}, \saph~\cite{Farber2012-vh}, \gorilla~\cite{Pelkonen2015-ko}, \qlikview~\cite{Qlik2011-ef}, \tableau~\cite{Kamkolkar2015-iq}, \monetdb~\cite{Boncz2002-yj}, \blink~\cite{Barber2012-xt}, and \sapnw~\cite{Lemke2010-is}. In-memory database systems are used where performance and low latency is a key design goal, and on systems that have no need for persistent storage \cite{Zicari2012-is}. Most of these systems are tuned for ad-hoc analysis and \bi~queries, which is enabled by column storage, compression, and parallelization that maximizes CPU utilization and obtains extreme throughput.

\mde, a discipline that aims to increase developer productivity by raising abstraction levels, has identified that there always will be a gap between the business problem and the implementation \cite{France2007-ae}. One of the main goals of \mdd~research is to create technologies that shield software developers from complexities of the underlying platform. To cope with these complexities, and close the gap between the business problem and the implementation, a deep understanding of the gap bridging process is required, an understanding that are gained through experimentation and accumulation of experience. One of the major advantages of \mde~is that we express models in in such way they are closer to the problem domain and less bound to the underlying implementation \cite{Selic2003-qa}. In general, \mdd-tools generate programs that are just as memory and performance efficient as hand-crafted programs, but there are occasional critical cases where \mde~are outperformed.

\genus~is one of the market players in \mde, and like most vendors, they aim to close the gap between business logic and implementation through abstractions and models. Their platform, \gap, uses generic software concepts on a higher level of abstraction than regular programming languages that are precise and non-ambigous \cite{noauthor_undated-qy}. These concepts, and the underlying implementation, have been refined and improved through years of trial and error on real customer use cases. However, since the main focus of the source code of the platform is readability and maintainability, no particular attention has been paid to how data is represented internally. This has resulted in a system with high memory usage, a system uncapable of handling large data sets, and a system with poor performance on operations that work on multiple data elements, such as joins and data aggregations.

Our main motivation for this research is two-fold. First, we have identified a critical case in \gap, and likely \mde~in general, which is memory consumption and handling and analysis of large datasets. We are motivated to seek out how this problem can be solved in the context of \mdd. Second, we observe that \gap~has many similarities with an in-memory database: Data is fetched from persistent data sources, manipulated and analyzed in-memory, before being persisted in the data sources again. We are, therefore, motivated to solve the problem using techniques used in state-of-the-art in-memory databases, and see whether such technology applies to a \mde-tool. Overall, our research is of interest because it increases versatility of \mdd-tools and the number of problems where \mde~can be applied.

\section{Goals and Research Question}
\label{sec:Goals and Research Question}
Based on our moouedotivation, we define the following goals:

\setlength{\leftskip}{1cm}

\textbf{G1: Reduce memory consumption in \gap~and increase the platforms ability to handle large datasets.}

\setlength{\leftskip}{0pt}

We set \textbf{G1} not only to improve \gap and tackle its performance challenges, but also to see how this problem can be generalized and solved in the context of \mde. By \textit{handling large datasets}, we mean operations and analyses that work from several thousands to several million elements at a time. \todo{Poor definition of handling large datasets?}In addition to \textbf{G1}, we also set a broader, but yet intertwined goal:

\setlength{\leftskip}{1cm}

\textbf{G2: Introduce new evidence that \mde~can benefit from in-memory database technologies.}

\setlength{\leftskip}{0pt}

As mentioned, \gap, and likely other \mde~tools, have many similarities with in-memory databases. We are curious to see whether techniques used in in-memory databases, mainly those optimized for analytical workloads, can be applied to the context of \mde. 

To reach \textbf{G1} and \textbf{G2}, we address the following research questions:

\setlength{\leftskip}{1cm}

\textbf{RQ1: How does storage techniques used in an in-memory, analytics-oriented database, like columns and compression, affect \gap's ability to handle large datasets, and how must the application be changed to accomodate and utilize the new storage format?} \todo{Needs work, but you get the point?}

\setlength{\leftskip}{0pt}

By answering \textbf{RQ1}, we hope to address \textbf{G1} directly by making changes in \gap~that increases its ability to handle large datasets. However, by using \gap~as a proof-of-concept, \textbf{G2} is also addressed by drawing general conclusions on the intersection of \mde~and in-memory database technology.

\subsection{Research Scope}
\label{sub:Research Scope}
It is important to know that the scope of this research has changed during the project execution. The initial direction of the research aimed directly towards the \bi~and analytics components in \gap, components which had the mentioned issues: High memory consumption and poor performance. Hence, the research started out with studying how these challenges could be overcome, with an emphasis on in-memory and read-only databases. However, as more insights were gained during the course of the research, it became apparent that our findings could be applied at a wider scope, thus came the idea to use the discovered techniques in the entire platform. Not only would this likely help \bi~and the analytical components of \gap, but also other parts of the platform.

\section{Methodology}
\label{sec:Methodology}
To reach our goals and to address our research question, we divide our research into two parts. The first part is a \textit{theoretical} literature review on in-memory databases and model-driven engineering, as well as an \textit{analysis} of \gap. The second part is a \textit{design and experiment} type research, where relevant findings from the literature review is applied to \gap. The modifications are tested by experiment, and conclusions are drawn based on the results.

\subsection{Literature Review and Initial Research}
\label{sub:Literature Review and Initial Research}
The main goal of the initial research and the literature review is to gain a deeper understanding of in-memory databases, \mde, and \gap. In Section \ref{sub:Research Scope}, we saw that the original scope of the research was to improve analytical capabilities in \gap, thus this analysis emphasizes techniques used in online analitycal processing (OLAP) databases. 

Most of the literature review was performed using a method known as \term{Snowballing}, which is convenient if the scope of the project is uncertain \cite{Ang2014-nm}. The Snowballing method is the process whereby you start with a few number of authoritative papers, and based on these you expand your list of readings by relevant work that the papers have cited. The identification of papers can also happen in the other direction, where you look for papers that have used the current one as a reference. Either way, this method is known to generate a large number of papers, so the researcher must be very strict and objective in which papers to read.

The initial papers, theses, and books used in this research were found in collaboration with department staff and regarded in-memory databases, columnar storage, and online analytical processing (OLAP) workloads. Both forward and backward searching was performed, and each paper was considered by reading the abstract, and conclusion and introduction if needed, to be put on the reading list. During the search process, we picked articles that could help us answer \textbf{RQ1}. When the field felt properly understood, we concluded the \term{Snowball} literature study. A similar process was used for background theory on \mde.

To gain a deeper understanding of \gap, a combination of technical briefs, lessons from \genus' employees, and source code analysis was used.  This part also included a study on \delphi, the programming language used to develop the \gap~core. \todo{requires rework?}

The results of this literature review and initial research are found in the background theory, Chapter \ref{chap:background}, and the chapter about \gap, Chapter \ref{chap:gap}.

\subsection{Design and Experiment}
\label{sub:Design and Experiment}
The main goal of the design and experiment part of this research is to directly answer \textbf{RQ1}; how does the topics discovered in the literature review apply to \gap, and which conclusions can be drawn for \mde~and database technology in general. According to France \ea, applying new techniques in a \mdd-tool requires systematic accumulation of experience \cite{France2007-ae}, and this is a measure to gain this experience.

More specifically we change the internal data representation in \gap, by implementing a column store, enhancing data formats, and apply compression. We also modify certain operations such that they utilize the new storage format. These techniques were learned from the literature study and aims to increase performance for analytical workloads. We run experiments on benchmarks for analytical databases to see whether such performance can be improved, but also do some limited testing to indicate whether other parts of the application are affected by the changes.

As the literature review yielded many promising techniques, but we were unsure of which techniques applied, we work in an iterative fashion. The first iterations focused on reducing memory consumption. First, we implement column store, then we enhance data representation by using primitive data types, and then we apply compression. During these stages, performance for different operations was monitored. At the fourth, and last iteration, we implement operations that utilize the new storage format. This iterative way of researching is presented in reflected in the main thesis, which consists of four main parts. The research yielded some blind roads, which are either presented along with the main content, or as a part of Chapter \ref{chap:misc}. 

%We focus primarily on three major tasks:
%\begin{itemize}
    %\item \bd~capabilities.
    %\item Batch operations.
    %\item Filtering tasks.
%\end{itemize}

% theoretic/analytic
% model/abstraction
% design/experiment
% 

\section{Deliverables and Contributions}
\label{sec:Deliverables and Contributions}
The main deliverable from this research is this report containing implementation details, test results, and discussion of our findings. The source code is not a part of this delivery, although some listings are provided within the report. The report has a practical approach, and instead of a thorough theoretical analysis of the results, it is meant to be a report showing the potential in the technology, inspiring further research and accumulation of experience.

This research contributes to an \textit{improved computer-based product}. \gap~is used as our main implementation platform and this product will benefit from the new techniques we have applied. We hope other \mdd-tools also profit from this research.

Second, and perhaps more important, is that we \textit{introduce new evidence} that \mde~can benefit from in-memory database technologies. We have been unable to identify studies taking this approach. This way, we \textit{re-interpret} the technology and apply it in a new context, and see whether object representation in \mde~should be reevaluated.

\section{Thesis Structure}
\label{sec:Thesis Structure}
% some examples in english matters
We use the two first implementation chapters to focus on reducing memory footprint, then the last one to improve the performance of certain operations.

We structure our thesis as following:
\begin{itemize}
  \item \textbf{Chapter 2} introduces background material relevant to this report, and introduces database technology and model-driven enginering. 
  \item \textbf{Chapter 3} outlines \gap, how it works, and challenges with the platform.
  \item \textbf{Chapters 4-7} are the implementation chapters. For each chapter, design changes are made, the changes are tested, results discussed and points at what should be done next
  \begin{itemize}
    \item \textbf{Chapter 4} shows how column store is implemented in \gap.
    \item \textbf{Chapter 5} shows how enhancement of data storage by introducing primitive data types can be used to reduce memory footprint in \gap.
    \item \textbf{Chapter 6} shows how compression techniques can be used to reduce memory consumption in \gap.
    \item \textbf{Chapter 7} elaborates on techniques that speeds up certain operations.
  \end{itemize}
  \item \textbf{Chapter 8} investigates other techiques explored in this research, which is UTF-8 encoding of strings and applying database statistics to select the correct storage format.
  \item \textbf{Chapter 9} discuss the results and implication of our findings, and sees it in a bigger picture.
  \item \textbf{Chapter 10} concludes this research and points at interesting directions for future work.
\end{itemize}



%However, several challenges with \mdd~exist, including performance \cite{Selic2003-qa}. However it is common knowledge that compiler optimization can outperform human creativity, and more recently: \mdd~tools generate code or systems that are faster and more reliable due to years of experimentation and experience. However, there are still problems left which leads developers to ditch the MDD approach all-together

%\genus, one of the market players in \mde, has one such system, \gap. This platform connects to a wide variety of data sources, and by defining data, logic, and GUI layers, a complete application model can be made. However, since this platform is made to be maintainable, no special attention has been given to how data is represented in the application implementation. This has resulted in a system with high memory usage, which causes paging, ucapable of handling large datasets, and poor performance on operations that work on multiple data elements. Performance challenges in \mde~is a well-known phenomena \cite{Selic2003-qa} \todo{rewrite, check the source}

%A new breed of \bi~products, which we refer to as \bd~products, has emerged recently, due to cheap, commodity hardware, based on in-memory technologies. Such products does not rely on preaggregated data, all results are calculated on-the-fly as needed, to help people answer their stream of questions and enable them to follow their own path to business insight. Such systems are enabled by in-memory technologies that uses column storage, compression and parallelization to maximize CPU utilization, hence improve user experience. Such systems decrease memory usage. \todo{rewrite, sources}

%\mde, a discipline that aims to increase developer productivity by raising abstraction levels, uses models, rules etc. to model an application instead of programming them. There will always be a gap between the business problem, and \mdd~aims to close that gap. Through years of experience, frameworks faithful to this discipline abstracts away complex problems, and apply optimizations. \todo{rewrite, sources}





%\subsection{Motivation}
%\label{sub:Motivation}
%One of \genus' main issues with \gap~is that it uses too much memory. The clients normally run in virtual environments with little RAM available. It is commonplace that a client runs out of memory and starts paging, which leads to drastically reduced performance. In addition, the hypothesis is that reduced memory consumption increases performance because memory management is expensione. In addition, there are several operations in \gap~that are slow, including \pn{Genus Discovery} load and filter operations.

%Therefore, we implement database techniques in order to:
%\begin{enumerate}
%    \item Reduce memory consumption to relax memory requirements for clients and improve performance by reducing expensive memory management.
%    \item See how a model-driven development tool benefits from database technology, and see which operations can be speeded up
%\end{enumerate}

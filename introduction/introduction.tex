\chapter{Introduction}
\label{chap:introduction}
In this chapter we explain the background information and motivation of this research project, and based on this, we state our goals and research questions. We then proceed to explain how we plan to answer the research questions by stating our methodology, deliverables, and contributions.
\clearpage

\section{Background and Motivation}
\label{sec:Background and Motivation}
A new breed of \bi~products, which we refer to as \bd~products, has emerged recently, due to cheap, commodity hardware, based on in-memory technologies. Such products does not rely on preaggregated data, all results are calculated on-the-fly as needed, to help people answer their stream of questions and enable them to follow their own path to business insight. Such systems are enabled by in-memory technologies that uses column storage, compression and parallelization to maximize CPU utilization, hence improve user experience. Such systems decrease memory usage. \todo{rewrite, sources}

\mde, a discipline that aims to increase developer productivity by raising abstraction levels, uses models, rules etc. to model an application instead of programming them. There will always be a gap between the business problem, and \mdd~aims to close that gap. Through years of experience, frameworks faithful to this discipline abstracts away complex problems, and apply optimizations. \todo{rewrite, sources}

\genus, one of the market players in \mde, has one such system, \gap. This platform connects to a wide variety of data sources, and by defining data, logic, and GUI layers, a complete application model can be made. However, since this platform is made to be maintainable, no special attention has been given to how data is represented in the application implementation. This has resulted in a system with high memory usage, which causes paging, ucapable of handling large datasets, and poor performance on operations that work on multiple data elements. Performance challenges in \mde~is a well-known phenomena \cite{Selic2003-qa} \todo{rewrite, check the source}

%Research has recently shown a keen interest in database technology for analytical workloads that enables high-performance analysis and on-the-fly aggregation recently, where the main motivation is \bi~products. Such products store data in main memory as compressed columns to maximize memory utilization and CPU throughput. \mde, a discipline that aims to increase developer productivity through the use of models on a higher level of abstraction, automates many of the complex programming tasks, like persistence and interoperability. One such product, \gap, has evolved over time and become a powerful and expressive tool for rapid application development. However, operations that process large amounts of data are slow, and the platform has a high memory footprint, mainly because no particular attention has been paid to storage format and structures in the source code. Based on the observation that \gap~has many similarities with an in-memory database, we are motivated to investigate if \gap~and \mdd~can benefit from database technology for analytical workloads.

% Paragraph about database technology and OLAP solutions.
%Database technology have had much progress in the last couple of years. The traditional OLTP systems with row stores are being supplemented with high-performance column stores for quick data analysis. Not only does column store improve performance aggregation performance, but it also reduces memory footprint as aggressive compression can be applied.


However, several challenges with \mdd~exist, including performance \cite{Selic2003-qa}. However it is common knowledge that compiler optimization can outperform human creativity, and more recently: \mdd~tools generate code or systems that are faster and more reliable due to years of experimentation and experience. However, there are still problems left which leads developers to ditch the MDD approach all-together.

One such problem is memory usage and tasks that affect a large amount of the data simultaneously. In this thesis, we explain and implement a column store in a model-driven development tool that will, without affecting the system developers, store data in highly optimized and compressed columns, such that memory usage can be improved. We apply technology from the recent database research and apply it to a development framework to allow for optimizations.


% Paragraph about the intersection of these two fields.
Database technology and column store is one such implementation that can be changed out and applied to the whole framework. We therefore implement column store within \gap to see which aspects of this technique are beneficial for \mdd.

% Paragraph of what we will investigate (GENUS specific)

% Paragraph about how we got the idea to the research
Our original research was focusing on business discovery optimalizations. However, we came up with the idea to use the techniques learnt from the database techonolgy dircectly as base structures within \gap.

\subsection{Motivation}
\label{sub:Motivation}
One of \genus' main issues with \gap~is that it uses too much memory. The clients normally run in virtual environments with little RAM available. It is commonplace that a client runs out of memory and starts paging, which leads to drastically reduced performance. In addition, the hypothesis is that reduced memory consumption increases performance because memory management is expensione. In addition, there are several operations in \gap~that are slow, including \pn{Genus Discovery} load and filter operations.

Therefore, we implement database techniques in order to:
\begin{enumerate}
    \item Reduce memory consumption to relax memory requirements for clients and improve performance by reducing expensive memory management.
    \item See how a model-driven development tool benefits from database technology, and see which operations can be speeded up
\end{enumerate}

\section{Problem Statement, Goals, and Research Questions}
\label{sec:Problem Statement, Goals, and Research Questions}
Based on our motivation, we define the goal of the project as following: 

\textbf{G1: Improve performance and reduce memory consumption in \gap~for all the major use cases in the framework by applying state-of-the-art database technology in the core.}

According to France \ea~, there is a large gap between the business problem and the actual implementation \cite{France2007-ae}. \mdd~aims to close the gap, or at least make it a little smaller. According to these authors, closing this gap requires a deep understanding of the problems which we do through experimentation and systematic acumulation of experience. We plan to gain exacly that experience in solving high memory usage and poor performance on certain operations. 

We focus primarily on three major tasks:
\begin{itemize}
    \item \bd~capabilities.
    \item Batch operations.
    \item Filtering tasks.
\end{itemize}

We define \textit{state-of-the-art etc.} as following:
\begin{itemize}
    \item It's super new and the best!
\end{itemize}

We have defined this goal because we know that for certain tasks \gap~are too slow and use too much memory.

As a step towards reaching \textbf{G1}, we address the following research question:

\textbf{RQ1: Which benefits will a column store give a model driven development framework?} 

For \textbf{RQ1}, not only do we study the benefits, we also try to define a set of operations that can be used by other parts of the application to create efficient code.

\textbf{RQ2: Which metadata should be used to decide on the storage format?} 

\section{Methodology}
\label{sec:Methodology}
The main part of the literature review was conducted using a method known as \term{Snowballing}, which is convenient if the scope of the project is uncertain \cite{Ang2014-nm}. The Snowballing method is the process whereby you start with a few number of authoritative papers, and based on these you expand your list of readings by relevant work that the papers have cited. The identification of papers can also happen in the other direction, where you look for papers that have used the current one as a reference. Either way, this method is known to generate a large number of papers, so the researcher must be very strict and objective in which papers to read.

The initial papers, theses, and books used in this research were found in collaboration with department staff and regarded in-memory databases, columnar storage, and online analytical processing (OLAP) workloads. Both forward and backward searching was performed, and each paper was considered by reading the abstract, and conclusion and introduction if needed, to be put on the reading list. During the search process, we picked articles that could help us reach \textbf{G1} and answer \textbf{RQ1}. Also, articles must have been published by a known digital library, journal, or conference. When the field felt properly understood, we concluded the \term{Snowball} literature study.

We implement column store in \gap and measure every change we do to get an overview over what affects the total performance.

% theoretic/analytic
% model/abstraction
% design/experiment

\section{Deliverables and Contributions}
\label{sec:Deliverables and Contributions}
The main deliverable from this research is this report containing implementation details and test results.

Our contributions of this research are two-fold. First, we plan on \textit{introduce new evidence} for the intersection of \mdd~and database technology, something we claim has not been done before. We see several papers on \mdd, but none of them have explicitly studied the storage format within such systems.

Secondly, we contribute to an \textit{improved computer-based product}. We hope the users of \gap~will be excited about the results.


\section{Thesis Structure}
\label{sec:Thesis Structure}
% some examples in english matters
We use the two first implementation chapters to focus on reducing memory footprint, then the last one to improve the performance of certain operations.

We structure our thesis as following:
\begin{itemize}
  \item \textbf{Chapter 2} introduces background material relevant to this report, and introduces database technology and model-driven enginering. 
  \item \textbf{Chapter 3} outlines \gap, how it works, and challenges with the platform.
  \item \textbf{Chapters 4-7} are the implementation chapters. For each chapter, design changes are made, the changes are tested, results discussed and points at what should be done next
  \begin{itemize}
    \item \textbf{Chapter 4} shows how column store is implemented in \gap.
    \item \textbf{Chapter 5} shows how enhancement of data storage by introducing primitive data types can be used to reduce memory footprint in \gap.
    \item \textbf{Chapter 6} shows how compression techniques can be used to reduce memory consumption in \gap.
    \item \textbf{Chapter 7} elaborates on techniques that speeds up certain operations.
  \end{itemize}
  \item \textbf{Chapter 8} investigates other techiques explored in this research, which is UTF-8 encoding of strings and applying database statistics to select the correct storage format.
  \item \textbf{Chapter 9} discuss the results and implication of our findings, and sees it in a bigger picture.
  \item \textbf{Chapter 10} concludes this research and points at interesting directions for future work.
\end{itemize}




\chapter{Conclusion}
\label{chap:Conclusion}
bla bla bla

\section{Evaluation}
\label{sec:Evaluation}
We are confident that \mdd~can learn a lot from the database world.

By implementing such storage in a model-driven development tool, we hope to increase the potential of the tool.

\section{Contributions}
\label{sec:Contributions}
We have successfully introduced new evidence for the performance benefits of combining \mdd~and database technology.

\section{Future Work}
\label{sec:Future Work}
There are still several issues that needs to be addressed. First, we have only briefly used the metadata to select the correct algorithm. There are several indications of which format is the correct one (code domain or object domain). One may even imagine a system that choses the correct storage format for different tasks.

For now, we have treated user level data and data structures differently. However, these aspects are conceptually the same. One might imagine an implementation where all the pointers also are stored in columns.

Last, but not least, we may look back at the issue that inspired the whole research: \bd. For now, \bd~capabilities have only been improved implicitly by reducing memory usage during runtime. In addition, we have improved load time by providing methods for generating required data arrays and indexes. However, most of the work still happens in a separate module without exploiting the data structures implemented in this research. Future work should implement joining, grouping, and aggregation directly within the columns, such that the proper techniques are used for the various implementations.

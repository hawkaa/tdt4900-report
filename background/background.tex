\chapter{Background Theory}
\label{chap:background}
This chapter introduce important background theory relevant to this research. First, we introduce \mde. Then, to introduce a use-case where handling and analyzis of large datasets is required, we give a brief introduction to \bi~and \bd. Last, we study \delphi, the programming language used in \gap~core.

\todo{Is this a good pitch on why we introduce bi and bd?}
\clearpage


\section{Model-Driven Engineering}
\label{sec:Model-Driven Engineering}
Since the introduction of computers, researchers have been working to raise the abstraction level at which software developers create programs \cite{Atkinson2003-wr}. One example of this is compilers and how they programmers no longer need to know assembly language for the different processor architectures. Now, object-oriented programming languages enable software developers to create own abstractions and solve more complex problems than ever. \mde~is a continuation of this trend, where the goal is to automate many of the complex, but routine, programming tasks. Such tasks include interoperability, distribution, and persistence. One of the key technological foundations of \mde~is support for visual modeling.

\mde~research has identified that there is a gap between the problem domain and the software implementation domain \cite{France2007-ae}. Thus, most research focuses on bridging this gap. This process requires systematic experimentation and accumulation of experience, and new techniques should strive to reduce unecessary complexities.

The ability to express models with concepts that are closer to the problem domain and less coupled to the underlying implementation is one of the main advantages with \mde~\cite{Selic2003-qa}. This makes programs easier to understand and maintain, as well as making program specification and requirements easier to communicate.  In addition, according to Atkinson \ea, \mde~has two more major advantages \cite{Atkinson2003-wr}:
\begin{itemize}
    \item In the short term, \mde~increase developer productivity and increase the amount of functionality an IT artifact can deliver.
    \item In the long term, \mdd~supporting infrastructures are more maintainable, and it takes a longer time before the artifact becomes obsolete.
\end{itemize}

A common discussion topic in the field of \mde~is the efficiency of the modeled programs. Criticism has pointed out that machines cannot optimize code better than a creative human being with clever tricks \cite{Selic2003-qa}. However, it is widely known that modern compilers outperform most system developers, and it does so more reliably. The same observation can be done for most \mdd~supporting infrastructures. Current tools generate programs that are within 5 to 15 percent effective as hand-crafted systems, both regarding memory consumption and performance. However, there are still critical cases where manually written code significantly outperforms \mde, which has often been used as an excuse to discard an \mdd~approach altogether.

\subsection{Models and Model Levels}
\label{sub:Models and Model Levels}
\afigure{img/uml-mof.png}{Traditional \mdd~with Unified Modeling Language (UML) and Meta-Object Factory (MOF) model levels. (Adapted from \cite{Atkinson2003-wr})}{fig:uml-mof}{0.7}
Central in \mde~is the use of models. According to Leppänen M, a model is a thing that is used to help the understanding of some other things \cite{Leppanen2006-ay}. A good model must appeal to our intuition, and it should reduce the amount of intellectual effort required to understand what the model represents \cite{Selic2003-qa}. In the field of \mdd, there are multiple model levels, where a model on a higher level describes/prescribes models on the next lower level. 

Leppänen M distinguish between instance models, type models and meta models \cite{Leppanen2006-ay}. An \textit{instance model} is a model which contains concepts that are instances of some other model, the \textit{type model}. Analogously, the \textit{type model} is an instance of the \textit{meta model}. The latter is used to enable understanding, communication, analysis and design of models. \textit{Model levels} are composed of models that comprise concepts on the same level. There are four model levels: instance level, type level, meta level, and meta-meta level.

Depending on the research field and what is being represented, the four model levels have different names and interpretation \cite{Leppanen2006-ay}. In information processing, the layers are:
\begin{itemize}
    \item \textbf{Information System (IS) layer}, typically represents day-to-day information processing actions in an organization, like order processing or inventory control. This layer represents changes in an application.
    \item \textbf{Information System Development (ISD) layer}, which is the layer where information systems are analyzed, designed, implemented, and tested. This layer contains descriptions of a specific application.
    \item \textbf{Method Engineering (ME) layer} is where techniques and procedures for the ISD layer are developed, selected, configured and customized. In this layer, new programming tools are created, for instance \mdd-architectures and new programming languages.
    \item \textbf{Research work (RW) layer}, involves research that aims to produce better methods and concepts for the method engineering layer.
\end{itemize}

Unified Modeling Language (UML) and Meta Object Facility (MOF) is the first generation of \mdd~infrastructure \cite{Atkinson2003-wr}, and is illustrated in Figure \ref{fig:uml-mof}. This generation has defined central components and terminology in \mde~research. However, this method has been criticized for lacking a clear description of how different layers relate to the real world. Also, it has a single-instance relationship between the model levels, which causes inconsistencies.

\afigure{img/linguistic-metamodeling.png}{Linguistic metamodeling. Linguistic relationships span across model levels while ontological relationships are contained within the same model layer. (Adapted from \cite{Atkinson2003-wr})}{fig:linguistic-metamodeling}{0.7}
To overcome the challenges with UML and MOF, other frameworks have been proposed. One such framework is \textit{linguistic metamodeling}, which separates linguistic relationships from ontological ones \cite{Atkinson2003-wr}. As seen in Figure \ref{fig:linguistic-metamodeling}, linguistic relationships span across model levels, while ontological relationships are typically on the same level. 

\subsection{Model-Driven Engineering in Practice: Mendix}
\label{sub:Model-Driven Engineering in Practice: Mendix}
Research on \mde~has resulted in several tools and methods. Some of these build on a sound, theoretical foundation, others take a more pragmatic and practical approach. Either way, the success of a tool taking a \mdd~approach is heavily dependent on how application development in these tools is done in practice \cite{Henkel2010-sy}. In this section, we study \textit{Mendix Business Modeler}, or \mendix, to see how this product has approached \mde. 

Applications in \mendix~are designed using three main models: An \textit{information structure model}, which defines the main data objects in the application, a \textit{microflow model}, which defines logic, and a \textit{form model}, which defines system user interface. Applications are designed in a modeling tool and deployed to a model repository, where they become available for end users.

\afigure{img/mendix-obj.png}{Class diagram in \mendix. (Adapted from \cite{Henkel2010-sy})}{fig:mendix-obj}{0.7}
The \textit{information structure model} defines the main data structure in an application, and is designed in an UML-like class diagram. As seen in Figure \ref{fig:mendix-obj}, this model does not only suport class definitions with properties, but also relations between objects. This model may also define validation rules and events which trigger on object creation, modification, and deletion.

\afigure{img/mendix-microflow.png}{A microflow in \mendix. (Adapted from \cite{Henkel2010-sy})}{fig:mendix-microflow}{0.8}
In \mendix, the \textit{microflow model} define complex logic in an application and allows the designer to extend the system with custom behaviour. Microflows may change objects, control how and when forms are displayed, call custom Java code, or integrate with external web services. Microflows are designed in a visual editor which we see in Figure \ref{fig:mendix-microflow}. 

\afigure{img/mendix-form.png}{Form designer in \mendix. (Adapted from \cite{Henkel2010-sy})}{fig:mendix-form}{1.0}
Forms and other graphical user elements are designed using a form editing tool, similar to those found in popular programming IDEs. The form editing tool is depicted in Figure \ref{fig:mendix-form}. \mendix~supports basic input fields, drop-down lists, tables, and more. A form can be associated with objects, other forms, and microflows. 

When the application is deployed to the model repository, end users can access the system using a web browser. When a web page is requested from the server, the HTML and JavaScript required showing the page is created and sent to the model. This interpreted approach is different from other \mde~tools, like OptimalJ, which relies heavily on code generation. \todo{Important content, but rewrite paragaph}

\gap~has many similarities with \mendix. We study \gap~in Chapter \ref{chap:gap}.

\section{Business Intelligence}
\label{sec:Business Intelligence}
\bi~is normally described as tools and techniques used to transform unstructured data into useful and meaningful information that can be used to support decisions \cite{Wikipedia_contributors2015-ag}. The goal is to allow for easy interpretation and gain insight in the data, such that businesses end up with a competitive market advantage. \bi~software can assist in making a wide range of business decisions, including strategic decisions as goals and product pricing or positioning, as well as operational decisions like priorities.

A challenge in \bi~can be information overflow. To know which information is needed in a \bi~application, decision makers must be aware of which types of decisions they should make, and have a model for each \cite{Ackoff1999-wk}. Since managers rarely fulfil the latter requirement, they add a safety factor and asks the IT department to provide everything. The result is information overload, and the much of the data is irrelevant. 

We introduce \bi~in this thesis to explain a common use-case where the ability to handle and analyze large datasets is required. \bi~reqirements in \gap~is one of the key motivations to this research. \todo{Should this pitch be used?}

\subsection{Data Warehouses}
\label{sub:Data Warehouses}
Companies' need for \bi~has traditionally been solved by data warehouses. These data warehouses extract, transform, and load data into structures that are suited for analytical queries and report generation.

The challenges with data warehouses are many, among them, the lack of flexibility. Reports are usually preconfigured and implemented by the IT department, which can result in lengthy reporting backlogs. Besides, the \qlikview~developers have pointed out some other drawbacks of traditional query-based \bi~tools \cite{Qlik2010-ya}:
\begin{itemize}
  \item Only small subsets of the main dataset are extracted at a time. These subsets are divorced from the data that was not included in the query.
  \item Each query represents a single piece of information, and the information gathered from individual queries are hard to combine.
  \item Traditional systems do not maintain relationships between queries. A query can be hard to formulate, and it is not always easy to know what to look for. Traditional \bi~applications do not let the user build queries step by step.
\end{itemize}

\subsection{Star and Snowflake Schema}
\label{sub:Star and Snowflake Schema}
Many \bi~applications and data warehouses usually access data that is organized in a \textit{star or snowflake schema} \cite{Barber2012-xt}. Distinct for such schemas is that they have a huge fact table, which can have millions or billions of rows, and smaller dimension tables, each representing some aspect of the fact rows (e.g. category, region, time). The fact table is connected to the dimension tables using foreign keys. A snowflake schema is an extension of the star schema, where one or more dimension tables can have relationships that further describe a dimension.

Star and snowflake schemas are normally used in \bi~applications because they are easier to optimize \cite{Lamb2012-kg}. The query optimizer creates efficient query plans by filtering and applying joins on the most highly selective dimensions first. Secondly, queries on star and snowflake schemas are easier to anticipate, such that indexes, materialized views, and/or denormalization can be applied to improve query efficiency \cite{Barber2012-xt}.

The disadvantage of using star and snowflake schemas for \bi~is the lack of flexibility. There are certain situations where there are more than one large fact table and situations where there is no clear distinction between fact and dimension tables.

\section{Business Discovery}
\label{sec:Business Discovery}

To overcome the challenges with traditional \bi~systems, a new type of products have emerged. We call these for \bd~products, a notion that was introduced by \textit{Qlik}~\cite{Qlik2014-vd}. \bd~products normally build on in-memory technologies and are fast, elegant, and end user intuitive solutions to analyse business data. Examples of such products are \powerpivot, \tableau, and \qlikview. We have only studied \qlikview~and \tableau~in this thesis.

\ffigure{img/qlik-hierarchy.png}{Comparison of a traditional reporting application and \qlikview. Traditional \bi~applications normally have predefined drill-down paths. \qlikview~allows the user decide where to start and end. (Adapted from \cite{Qlik2014-vd})}{fig:qlik-hierarchy}

\bd~products allow users to follow their "information scent" or "train of thought" when navigating through the data \cite{Kamkolkar2015-iq, Qlik2014-vd}. As seen in Figure \ref{fig:qlik-hierarchy}, there are no prespecified drill-down patterns, and users decide where to start and end. In these applicaitons, grouping, joining, and calculations are performed on-the-fly. High-performance, in-memory technologies are used to enable such functionality. 

In the next sections, we explain how a typical \bd~application works by using \qlikview~primarily as an example. \tableau~and \powerpivot~work similarly.

\subsection{Data Import}
\label{sub:Data Import}
Data is loaded into a \qlikview~document using a data import script that connects the application with data sources like databases or files. The script uses an SQL-like syntax that lets the user specify names of fields and tables that are used in the data analysis. When the data import script is run, data is fetched from the sources and put into the \qlikview~in-memory engine such that data can be queried efficiently. No queries are proxied to the underlying data sources, \qlikview~manage all queries internally.

\afigure{img/qlikview-association.png}{Four tables: Countries, customers, transactions, and memberships. The fields \textit{Country} and \textit{CustomerID} associate the tables. (Adapted from \cite{QlikTech_International_AB2011-ov})}{fig:qlikview-association}{0.7}

The data import script will regularly include more than one table. Multiple tables are \textit{associated} if they have fields with the same name, as seen in Figure \ref{fig:qlikview-association}. Internally, there can be only one data connector between a pair of tables, so if multiple tables refer to a single table, a mechanism (either \textit{synthetic keys} or \textit{loose coupling}) must be applied to logically duplicate the table. This restriction ensures that there exists no more than one possible join path between any pairs of tables \cite{noauthor_undated-js}.

In the data import step, data might be preprocessed and transformed. First, data can be aggregated, for instance by summarizing or grouping, before being loaded into \qlikview. Second, data might be filtered. Both these techniques reduce data volumes. Lastly, data can be denormalized in the import script, i.e. pre-joining tables before import. Denormalizing can be done to lower the number of tables in the data extract and, by doing so, reduce application complexity.

\subsection{User Interface}
\label{sub:User Interface}
\ffigure{img/qlik-panel.png}{\qlikview~dashboard with various GUI elements, like lists and charts. Selections are green, matched data is white, and unrelated data is grey. (Adapted from \cite{Qlik2014-vd})}{fig:qlik-panel}
Users interact with the \bd~application through a reporting dashboard, as seen in Figure~\ref{fig:qlik-panel}. Requirements for such panel are typically \cite{Qlik2014-vd}:
\begin{itemize}
  \item Clicking field values in list boxes.
  \item Lassoing data in charts, graphs, and maps.
  \item Manipulating sliders.
  \item Choosing dates in calendars.
  \item Cycling through different chart types.
\end{itemize}

Users navigate through the data by making selections in the user interface. When a selection is made, the item is made green, as seen in Figure \ref{fig:qlik-panel}. The current selection is also known as the \term{application state}. Upon selection, the rest of the elements in the panel are updated based on the new application state; aggregations are recalculated, and graphs and lists are updated. \qlikview~colors matched elements white and unmatched elements gray. Dashboards are typically available from different devices, including desktop comptures, tablets, and mobile phones.

\subsection{Business Discovery and Queries}
\label{sub:Business Discovery and QueriesL}

We see that \bd~products interact with the data using selections and filters in a reporting panel. This technique is different from database systems designed for OLAP workloads, which normally use a \textit{query-based} interface, usually SQL. Still, \bd~products need to provide most functionality that SQL databases do, like row listing, filtering, joining, grouping and aggregation, and sorting.

Queries, or requests, in a \bd~application have certain characteristics. First of all, they cannot be anticipated, more specifically, they are \textit{ad-hoc}. Secondly, there is a limit to the number of returned rows. Users are interested in queries that can be analyzed quickly, so we do not expect a \bd~application to return thousands of rows as a result from a single table \cite{Ferrari2012-hm}. Also, the user interface has limitations in how much data that can be displayed at the same time. 

\section{Delphi Programming Language}
\label{sec:Delphi Programming Language}
The core \gap~source code is written in \delphi. It is, therefore, important to understand how this language works. \delphi~is both a programming language and an integrated development environment \cite{Wikipedia_contributors2016-jk}, however, we will only consider the programming language in this section.

\delphi~is a strongly typed, high-level, object oriented programming language \cite{noauthor_undated-cn, Wikipedia_contributors2016-jk}. The language supports polymorphism and interfaces, unicode, inline assembly, generic programming, pointer arithmetic, function overloading, and more. Simple preprocessing directives are allowed, but there is no support for macros. The \delphi~compiler is efficient and compiles source code into native binaries for wide variety of platforms, including \textit{Windows} 32- and 64-bit architectures and \textit{OSX} 32-bit architecture.

A program written in \delphi~is composed of different source code units, with two mandatory parts: The \textit{interface}, much like a header in \textit{C}, which declares constants, types, variables, and function signatures, and the \textit{implementation}, which contains the actual executable code \cite{noauthor_undated-pl}. Types, variables, and functions might also be declared in the implementation part. However, these will not be made accessible outside the unit. A unit might also specify an initialization and finalization code chunk to run on unit load and unload respectively. Each unit specifies its dependencies to other units, and the dependency graph is created automatically without using makefiles or similar mechanisms.

\subsection{Delphi Types}
\label{sub:Delphi Types}
\delphi~supports a wide range of data types. In the language taxonomy, there is a total of seven different type categories. In this section, we elaborate on the three most important categories.

\subsubsection{Simple Types}
\label{ssub:Simple Types}
Simple Types in \delphi, which are sometimes refered to as primitive data types in other programming languages, are divided into two categories: Ordinal and real \cite{noauthor_undated-st}. Simple data types are byte aligned, which means no data type takes less than 1 byte.

Ordinal types, which define ordered sets of values, include integer, character, Boolean, enumerated, and subrange types. The different types have different sizes and ranges, for instance, integers can be represented by 32 or 64 bits.

A real data type defines a set of numbers that can be represented with a floating-point notation. Types include single (4 bytes) and double (8 bytes) precision floating point numbers, as well as fixed-point data type for currencies.

\subsubsection{String Types}
\label{ssub:String Types}
\afigure{img/delphi-strings.png}{String structure in \delphi. All strings are heap allocated, although string pointers may exist both in the stack and on the heap.}{fig:delphi-strings}{1.0}
\delphi~has string support built into the language. Strings are reference counted and dynamically allocated on the heap by the compiler \cite{Wikipedia_contributors2016-jk}. A string variable is a pointer to a structure that contains a 32-bit length indicator, a 32-bit reference count, a 16-bit data length indicating the number of bytes per character, and a 16-bit code page \cite{noauthor_undated-cp}. This is depicted in Figure \ref{fig:delphi-strings}. There is copy-on-write semantics if one variable that references the string changes the contents of the string.

There are two major string types in \delphi: \cn{AnsiString} and \cn{UnicodeString}. \cn{UnicodeString}, which is the default, supports both UTF-8 and UTF-16 encodings, where UTF-16 is default on the \textit{Windows} platform. UTF-16 stores each character with two or four bytes, while UTF-8 stores the first 128 ASCII characters with only one byte. Conversions between string types and encodings happen implicitly.

\subsubsection{Structured Types}
\label{ssub:Structured Types}
For representing more complex data structures, \delphi~provides a type category named \textit{structured types}. This category include set, array, record, file, and class types \cite{noauthor_undated-vu}.

\delphi~has two main array structures: Static and dynamic. Static arrays are allocated with a certain length, and will take up as much memory as the assigned length, even if values are not assigned. Dynamic arrays are more versatile; they are reallocated when you assign a value that has not yet been assigned. Dynamic arrays might also be reallocated using the \fn{SetLength} function. Like strings, dynamic arrays are reference counted, but they do not have compy-on-write semantics.

The \cn{record} type commonly represent a heterogenous set of values. This type has value semantics, i.e. they are not boxed types, which means data is copied on assignment and passed by value as function arguments. Hence, records can both be allocated on the stack and on the heap. Records are byte-aligned by default.

Classes in \delphi~are dynamically allocated blocks of memory whose structure is determined by the class definition \cite{noauthor_undated-un}. Class instances are allocated and deallocated by the memory manager by calling the constructor and destructor respectively. Values are stored in the same order which they are declared in the source code. In addition to the variables, the first 8 bytes of an object points to a virtual method table. Variables of class types are 64-bit pointers and can hold the value \cn{nil}.

\subsection{Program Control}
\label{sub:Program Control}
\subsection{Memory Management}
\label{sub:Memory Management}
In \delphi, the memory manager is replacable and dependent on which platform the code is compiled for \cite{noauthor_undated-ys}. Programs compiled for \textit{Windows}, which is the case for \gap, use the \textit{FastMM} memory manager. \textit{FastMM} is optimized for programs that allocate small to medium size blocks and it anticipates future block reallocations which reduces address space fragmentation. Block sizes are rounded up to the nearest 16 bytes. 


\subsection{Calling Conventions and Inlining}
\label{sub:Calling Conventions and Inlining}
Parameters are transfered to functions in registers or on the program stack depending on the calling convention of the function and the type of the parameters \cite{noauthor_undated-xi}. Simple data types are passed by value (unless the \cn{var} keyword is specified), while most other values are passed by reference. 

The default calling convention in \delphi~is \textit{register} which uses up to three registers to pass parameters. The remaining parameters, if any, are passed on the program stack. Hence, function calls with less than four parameters are more efficient, since no state must be written to memory.

and some simple functions
, and the stack 

Generally, simple data types are passed in regi

Value and constant parameters sent to a function is either passed by reference or by value, depending on the size of the variable \cite{noauthor_undated-ae}.

The default calling convention is \texttt{register}, which uses three registers and puts the rest on the stack \cite{noauthor_undated-xi}. This is most efficient, because it does not need to create a stack frame on every function invocation. In addition, all properties contain either none, one, or two parameters, hence are quite efficient. 

Value and constant parameters to a method call is either passed by reference or by value, depending on the size of the variable \cite{noauthor_undated-ae}. By default, the \fn{register} calling convention is used, which means up to three parameters can be stored in registers. Some types, including doubles and 64-bit integers, are always pushed on the stack.

The compiler supports inlining of methods using the \texttt{inline} directive. However, this does not work for virtual methods, across package boundaries \cite{noauthor_undated-rx} \todo{finne mer her}


\subsection{Standard library}
\label{sub:Standard library}
Some built-in classes in the standard library include:
\begin{itemize}
    \item \cn{TBits} is a built in type that represents a bitmap. It works like an array of booleans, except that it is optimized for speed using assemply code. In addition, it contains an utility method \fn{OpenBit}, which finds the first 0, or the open bit, in the bitmap.
    \item \cn{TArray} is the generic implementation of \cn{array of *}. It is ref-counted, dynamically allocated using the \texttt{SetLength} function. Its performance is studied in Appendix \ref{app:array-performance}.
    \item \cn{TList} and \cn{TObjectList} are wrappers for \cn{TArray} that automatically allocates enough space. \cn{TObjectList} is like \cn{TList}, but it assumes ownership over objects and automatically frees objects that are removed from the list.
\end{itemize}


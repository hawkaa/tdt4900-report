{\Huge Preface}
\vspace{1cm}
% The preface includes the facts - what type of project, where it is conducted, who supervised and any acknowledgements.
% http://www.writersandeditors.com/preface__foreword__or_introduction__57375.htm

This Master's Thesis is the final deliverable of the Computer Science Program at the Department of Informatics, Norwegian University of Science and Technology. Master's research at this program consists of two parts: A pre-study, due December 2015, and the main thesis, due June 2016. The research is a collaboration between NTNU and Genus AS. The research is executed by Håkon Åmdal, supervised by Svein Erik Bratsberg at the Department of Informatics, NTNU, and Einar Bleikvin at Genus AS.

The pre-study, due December 2015, focused on \bi~related performance improvements. The ultimate goal was to improve ad-hoc analytical abilities in \gap~by applying state-of-the-art in-memory database technology. This research resulted in a literature review and a report enumerating the most important aspects when engineering such system.

This thesis builds on the pre-study, but with a broader perspective: It inspired us to reevaluate how \gap~and other frameworks for \mde~should handle data internally, and see whether the storage formats and techniques from in-memory database technologies could be applied. The background theory in this repart, as well as certain parts of the introduction, is, therefore, adapted from the pre-study.

\vspace{1cm}

{\Large Acknowledgements}
% Read academic writing MIT for inpsiration


First of all, I would like to thank \genus~for proposing an interesting and challenging research project. It has been compelling working with storage representation and optimizations in \gap. Even more important to me, is that I have been working on a real business case and that my research has contributed towards a better product for \genus' customers.

It is my pleasure to express my gratitude to \textit{Einar Bleikvin}, my co-supervisor at \genus. He has during the project period always shown interest in my work, been a great discussion partner, and contributed with his years of experience in programming, \delphi, and \gap. Most important is that he has trusted me in working on central components in \gap, and helped me fix numerous bugs and memory leaks.

\textit{Johnny Troset Andersen} deserves recognition for his dedication to this research. Johnny has always provided be with the resources I need to execute the project, which includes access to key personnel within \genus~and relevant literature. Last, but not least, he has always provided me with a desk at \genus' office and made me feel like home.

I am grateful for the support I have received from \textit{Thomas Hedén}. Thomas has helped me establish benchmarks, access customer data, and maintain test environments. Ha has been a key contributor in report quality assurance, regarding language, structure, and content.

Other \genus~employees that deserve recognition include \textit{Bernt Almlid} for assistance with \gap~multitenancy and architecture, \textit{Ole Anders Bøe} for maintenance of benchmark environments, \textit{Petter Bergfjord} for valuable feedback on the report, and finally \textit{Martin Børke} for the input on the history of \bi. \todo{Add more here after feedback}

I would like to thank \textit{Svein Erik Bratsberg} for being my supervisor on this project. Svein Erik has kept me on track for the project period and has always held his office door open for me. He has provided me with key literature for this project, and provided valuable feedback on the project report. Svein Erik has been an irreplaceable asset for this project, especially in the final weeks before deadline.

Lastly, I am extremely grateful for \textit{Marte Løge}'s support.

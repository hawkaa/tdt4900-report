{\Huge Sammendrag}
\vspace{1cm}

Forskning har i den siste tiden vist stor interesse for databaseteknologi som muliggjør analyse av data med høy ytelse og umiddelbare beregninger, hvor hovedomotivasjonen for denne forskningen er produkter innen \textit{Business Intelligence} og \textit{Business Discovery}. Slik teknologi lagrer data i RAM på et komprimert kolonneformat for å få høy prosessorytelse og utnyttelse av minne. \textit{Modelldreven Utvikling} er en disiplin som forsøker å øke utviklerproduktivitet ved hjelp av modeller på et høyere abstraksjonsnivå, samt og automatisere mange komplekse programmeringsoppgaver, som datapersistens og interopabilitet. Et produkt innen denne disiplinen, \gap, har over tid utviklet seg til å bli et kraftig verktøy for rask applikasjonsutvikling. Likevel tar operasjoner som analyserer og prosseserer store datamengder lang tid, og platformen bruker mye minne, hovedsakelig fordi det aldri har blitt fokusert på intern datarepresentasjon. Siden \gap~har svært mange likhetstrekk med en minne-basert database, er vi motivert til å utforske om utfordringene i \gap~kan bli løst med teknologi benyttet i databaser optimalisert for leseytelse.

I denne oppgaven forbedrer vi datarepresentasjon, implementerer kolonnelagring med \textit{dictionary encoding} og \textit{bitpacking} i \gap~for å redusere minneforbruk og forbedre platformens egenskaper til å behandle og analysere store datamengder. I tillegg identifiserer vi kjerneoperasjoner som kan dra utnytte av det lagringsformatet, slik som join- og filtreringsoperasjoner. Vi tester implementasjonen vår med X mens vi i tillegg overvåker at skriveytelse ikke blir påvirket negativt.

For \gap~ser vi at kolonnelagring med \textit{dictionary encoding}, \textit{bitpacking} og nullpekerkompresjon reduserer minneforbuket med 67 \% i våre tester, i tillegg til at lastetiden reduseres med 36 \%. Operasjoner i platformen som har blitt omskrevet til å ta utnytte av det nye lagringsformatet ser en ytelsesforbedring på én, to, eller tre størrelsesordener. Den nye implementasjonen øker ikke skriveytelse signifikant. Vi har, ved å bruke \gap~som et eksempel, vist hvordan teknikker benyttet av databaser optimalisert for leseytelse kan benyttes i \textit{Modelldreven Utvikling} for å forbedre støtte for store datamengder, noe som igjen øker allidigheten til rammeverk for \textit{Modelldreven Utvikling}. 
